\subsection{Individual Project Report for Alexander Albert}
\par \underline{ECTS:} 6
\par \underline{Hours spent so far:} 148 
\par \underline{Short description of the tasks done so far:}
\par After joining the Rocket Science team for the second iteration I started with rewriting the questionnaire which had to be installed at the beacons together with Alexandra. According to the experience from different courses of psychology and computer science courses with empirical background we  paraphrased it so that it at least looks like a real questionnaire even if it not measures anything at all.
\par Additional to the rewriting of the questionnaire in english I translated it to german language so that we can offer both variants over the beacons right now.
\par Subsequently I started on a new simulation according to the goal of the rocket science group. First we thought about rebuilding the first simulation according to the new goal, but we decided to create a new one very fast. The old simulation considered the whole campus respectively the whole city but for that new goal we needed a more fine granular view to the moving of the students. 
Afterwords I started with building the new setup in NetLogo. We decided to simulate just the Erba building so I started with finding a areal view of the Erba building so that a user has a better visual understanding of wat going on during the simulation. After that I created all necessary rooms. We decided to split the rooms in arrival areas (where you can enter the campus), entrances (where you can enter or leave the building), lecture rooms (the two big lecture rooms in the ground level and all others summarized to lecture rooms in the seminar tower and the office area) and sitting rooms (cafeteria, library and outside in the patio). Next step was to adapt the students turtles to the new requirements. So I introduced new variables for the phones of the students and their behavior with it. After that I changed the input data of the first part of the project so that it fits to the new simulation. Additionally I set up the new movement of the students so that they e.g. walk over a entrance if they are outside of the building and have to go inside. The simulation was now set up as far that everything works properly for the first step.
\par Next step was to generate some realistic input data for the simulation. I got xml files with all rooms and lectures that take place at the at the erba building from the univis. I also tried to get some data from the cafeteria sales from the Studentenwerk, but it arrived just a day right after our final presentation. I builded a Java tool to convert the xml data from the univis to a format that our simulation can work with. I faced some difficulties with that, mainly because of the structure and the incompleteness of the data out of the univis, and so I spend a lot of hours with that.
\par Finally I reworked the simulation tool to work properly with the new data. So we finally have a good approach to what in fact going on at the erba building with our simulation and can easy simulate what will happen if things change in different ways by tuning the created controls in the simulation interface.