\section{Introduction}
This report documents our project work about researching student mobility with beacon technology and provides a detailed description about our approach.

\textbf{How is Student Mobility defined in our context?}
\par Over a considerable span of time, often it has been observed that students of the Otto-Friedrich University of Bamberg need to mobilize from one university campus to another. The campuses are scattered around the city and students only have a short span of time in between their classes to move from one to another. In this study patterns shall be figured out which could help:
\begin{itemize}
	\item Taking initiatives for the convenience of students.
	\item Making situation oriented predictions e.g. predict vehicle type on weather conditions.
\end{itemize}
\begin{figure}[H]
	\centering
	%\includegraphics[height = 80mm]{./images/motivation}
	\caption{Motivation for student mobility}
	\label{figure1:introduction_motivaton_mobility}
\end{figure}
There can be many approaches to study mobility such as
\begin{itemize}
	\item Observation and Survey
	\item Stationary/Mobile Sensors/Mobile Applications
	\item Simulations/Statistics  
\end{itemize}
In the upcoming sections we document our approach of exploring and utilizing the potentials of the beacon technology and Physical Web for studying student mobility in Bamberg.

\subsection{Motivation} 
This is page is dedicated for motivation.



\subsection{Terms and Concepts} 
This page is dedicated for terms and con concepts.
\subsection{Goal Definition} 
This page is dedicated for goal definition.
