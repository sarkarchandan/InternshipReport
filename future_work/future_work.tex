\section{Future Work}

Further studies about beacons can give more insight about possible improvements in the detection rate. 
\par On the one side it can be tested how the awareness and the interest for beacons and the Physical Web can be raised to get more detections. On the other side the simulation can be refined to get more accurate data about future scenarios.

\subsection{Beacon and Physical Web Related Future Work}
Based on the outcome we have for our study so far, we would like to mention following future steps that could be incorporated. If these steps are taken we expect a better outcome for the mobility study in future.

\paragraph{Grab More People Attention:}
We suspect that our attempt in the second phase of beacon deployment and data collection was unable to grab sufficient attention of the students. We posted a notice in the notice boards of each of the designated locations but its impact was not reflected in the outcome.

\par We propose to put more signs in the designated locations where the beacons are placed. Signs should  be of nature that are easily visible and can grab people's attention.

\paragraph{Make More Content Available:}
In our study we could put forward just a survey for the students to fill in. With this we think that there is a lack of motivational factor for students to interact with th beacons. Although we have asked in our survey what sort of information or content would be useful or helpful for the students, if we could pro actively make some useful or exciting content on the beacons, it could add to the motivational factor for students to interact with the beacons and the url they are emitting.

\par Such interactive content could be the menus in the cafeteria of designated locations or schedules/reminders of an event for students being organized locally. Overall, we believe that since Physical Web with Eddystone compliant BLE beacons is a niche approach in Internet Of Things, we need to create more awareness in potential users by grabbing their attention towards it.

\subsection{Simulation Future Work}

Based on the results of the simulation and the work with Netlogo there are different scenarios conceivable for the future where could be worked on. 

\paragraph{Additional Data}

First there is to mention that there could be more actual data to import in Netlogo. The more data there is available to import the results of the simulation will be more accurate. In the following there will be discussed what data would be useful to import to the simulation and how this data could be generated.

\par There should be collected data, how many people are actually at the Erba to a given time. With this data the simulation output would be more accurate because better assumptions of how many people could detect the installed beacons could be made. This data could be generated by occasional counting how many people will enter and leave the building during the entrances. A more interesting way of counting people would be to install overhead cameras next to all entrances and count the incoming and outgoing outgoing people to get a average value over time.

\par According to the goal of the project to learn something about student mobility, there should be data added for the universities employees. So the simulation can make a difference of how many employees and how many students will interact with a beacon and the results will be more accurate. Maybe the data about the employees of the university could be get by the dean of the WIAI to get information of how many people are employed at the faculty. It also could be checked if the university leadership could provide this data, because not all employees at the erba building are members of the WIAI faculty. If no of this data could be provided, because of reasons of privacy or similar there could be made an approximation of how many people which have an univis page have there office at the erba building.

\par The data that could be generated out of the univis is a very good approximation compared with the estimated values of the beginning of the project. But compared with the actual numbers they are not accurate at all. Again compared to other parts of the university the data for the erba building is very groomed, thats why we decided to use the erba building for our example, but it could be better. A good way of getting more accurate data is to encourage all lecturers to make their univis data as accurate as possible. For example by providing all times and rooms of a lecture correctly, accurate assumptions of the expected number of participants or comply the default structure of the univis to make it better readable by machines.

\par Additional to the univis data there could be more data collected by other virtual systems of the university. For example could made assumptions which courses will be visited by students together if there would be statistics from the FlexNow or the Virtual Campus.

\par The Studentenwerk W{\"u}rzburg can provide statistics of how many people eat every day in their cafeterias and cafeterias. They can distinguish between how many students, employees and external people eat or buy at their facilities. This data could be used to make a better assumption of how many people visit the cafeterias or cafeterias and how many will detect or interact the beacons.

\par Maybe the libraries could also provide according data like the Studentenwerk, which could be used also to make the simulation more accurate

\par A survey could be created and rolled out to the students about what phones respectively what mobile operating system they are using. There could be found how many people actually have their bluetooth always active, location services always active or if they use iOS how many students have installed an app to detect physical web beacons and how often they interact with this apps.

\paragraph{Transfer To Other Buildings}

Once the simulation is set up for the erba building it could easily transferred to other buildings of the university, like the feki or the Markushaus. All what has to be done is to switch the background image, adjust the rooms, switch the import data for the courses and rooms and adjust the parameters in the simulations interface.
The data converter, which is added to the simulator and converts da data from the univis to a format that NetLogo could read, has to be adjusted too. Especially the translation of the rooms has to be reworked for that.

\paragraph{Improve The Simulation }

To make the simulation more accurate to the reality there could be done some improvements. 

\par Currently students that want to enter the building from outside or go outside from inside move over any entrance which is randomly chosen. Maybe there is a way to make the students go over the nearest entrance according to there next target. Also there is no difference in going inside of the seminar tower and the main building. maybe this could be changed, because actually students could choose the entrance of the seminar tower even if they want to go go the cafeteria or library.

\par At the moment there are the two big lecture rooms, the seminar tower and the main building respectively at one will be simulated. Maybe it makes sense to simulate more rooms to get a more fine granular output. This makes no sense if the only beacons that are installed in the cafeteria and the big lecture room, but if there are more beacons of different floors and maybe in different rooms of the building this could be added to the simulation.

\par Students actually just move during breaks. This makes sense for students that are in a lecture room, but not for that who sitting in the cafeteria or library. Maybe there could be made a difference between students which are in a lecture and them sitting around in the building.

\par Presently there where generated an given amount of additional students which are coming to the university even if they have no lecture at this moment. This leads to a good approximation of how many students actually are at the building but if there are no students at all at the erba to a given time there will be no additional students will be generated. Maybe it would be better to make this not dependent to the number of students in lectures, but to the time of the day. But additional data would be necessary to have a good assumption of how many people actually are at the building.