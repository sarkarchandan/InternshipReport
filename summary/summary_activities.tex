\subsection{Key Concepts and Activities}
This subsection will describe the key concepts developed during the internship and the activities performed.

\begin{itemize}
	\item Object Oriented Design Patterns
	\item Delegate Pattern
	\item iBeacon Standard and Ranging/Positioning
	\item Object Oriented Programming with Swift, Objective C
	\item Cross platform research with C++ 
	\item Test Driven Development
\end{itemize}

\subsubsection{Object Oriented Programming and Design Patterns}
M. D. Smith et al. \cite{smith2011object} have described the object oriented design paradigm as the specification of task to be performed in terms of the associated objects and properties/behaviors of the objects. They have further illustrated an object as an instantiated entity having some operations it is capable to respond. It has state which could be impacted by the associated operations. Objects can invoke operations of other objects by passing messages. Smith et al. \cite{smith2011object} has also described a class as an interface specifying the properties and behaviors of an object. Properties defined inside the class determines the state of the object when instantiated. OOP paradigm provides abstraction and encapsulation contributing to security and modularity the real power of re-usability comes in the form of inheritance. OOP has contributed in developing maintainable code for many large and sophisticated software applications. 
\par Despite the capabilities that OOP has offered, modern software development can not be realized without adherence appropriate architecture and design paradigm which needs us to adhere to certain disciplines while writing software application code. Erich Gamma et al. \cite{gamma1995design}  have mentioned that design patterns are the mechanism to identify, name and abstract away common themes in object oriented designs. They are generally based on the intent behind the design and they identify the collaboration, rolls and responsibilities of different objects in building a software application or solution. A very common frequently used example in mobile applications demographic is Model View Controller(MVC) design pattern which is often referred from the development of  Smalltalk-80 programming environment as illustrated in the work of Krasner et al. \cite{krasner1988description}. The key focus of the MVC design patterns is to separate the functional units of an application for modularity and easier maintenance. It separates the class instances encapsulating data and operations related to the application domain as \texttt{Model}, the presentation and display of the application state as \texttt{View} and user interaction and response with the model and the view as \texttt{Controller}. MVC is one of the popular design patterns used in developing standalone mobile applications. While there are many popular design patterns, in my internship I have worked with three popular object oriented design patterns as briefly illustrated in the table \ref{table:oop_design_patterns} below:

\begin{longtable}{|| c | p{.70\textwidth} |} 
	\hline
	\textbf{Design Patterns} & \textbf{Specification} \\ \hline
	\textbf{Mediator} & 
	Mediator Description\\ \hline
	\textbf{Strategy} &
	Strategy Description\\ \hline
	\textbf{Delegate} & 
	Delegate Description\\ \hline
	\caption{Object Oriented Design Patterns dealt with}
	\label{table:oop_design_patterns}
\end{longtable}