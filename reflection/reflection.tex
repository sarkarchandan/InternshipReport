\section{Reflection}
In this section I would like to reflect my views on the concepts that I have learned and the activities that I was engaged with. I would like to make it concise and to the point for three distinct areas that I have mentioned in the previous section.

\par I have learned the object oriented programing as part of my degree program and used the knowledge mostly in academic domain on delivering the assignment for various subjects of my curriculum. I had no prior experience on how to employ the object oriented methodology in a real world projects. Although I have theoretically studied the object oriented design patterns prior to starting my internship, I never before realized how to effectively use them in programming and how they make a difference between usual code that accomplishes the task and a maintainable modular and cleaner code. At the start of my internship I have worked on various smaller practice projects without using any design strategy and after learning about the design patterns, I have recreated the same projects and clearly understood the impact of clean architecture and how they help software developers to maintain a code-base where they can easily integrate a new feature or deprecate some older ones without sacrificing existing capabilities of the code. For instance Favendo Commander framework is a careful orchestration of independent modules which interact with each other in order to create complete solution. Without a mediator design pattern implementation in the heart of this orchestration,  each module would have called functions of several other modules directly through respective interfaces. It would have accomplished the task but at the same time would have created dense interconnections bound to become unmanageable beyond certain point when complexity grows and new features are implemented. By introducing mediator object in the middle we have achieved loose coupling and maintainability. We are now capable to add or deprecate a functionality without breaking existing once. Similarly using strategy design pattern made it easier for us to incorporate new algorithms to achieve a given task quickly and in much cleaner way.

\par Adhering to the Test Driven Development discipline for the first time while creating new a framework was difficult for me. I was more familiar with the practice of  writing code implementation first. It was difficult for me to write the test first knowing that it will lead to compilation error and instinctively I wanted to avoid that. Slowly, with practice and influence from my supervisors I tried to make myself familiar with this new paradigm. I slowly realized that the practice of writing tests first let's us rationalize our thought process from a broader perspective. We get to lay a solid foundation for our API layer with deciding what we want our framework to offer as public methods. Then we go on and define the implementation layer writing function code, slowly getting rid of the compilation errors and finally having the tests succeeded. This practice allows us to achieve a bug-free code for a framework from ground up. Moreover test cases function as a scratch pad for our thought process and intention behind the framework. If we realize that we want to made some modification in the API we can again start at refactoring our tests resulting in failure and subsequently going into implementation layer and making necessary changes leading to succeeding tests again. Test Driven Development thus harmonizes the development process to a greater extent.

\par Studying and working with Beacon Ranging and Positioning concepts opened an entirely new frontier of expertise before me. I have learned to work with entirely new technology and think about applications of new nature. I have learned about associated challenges like smoothing the beacon signal with techniques like Kalman Filter in order to derive usable signal data that we can use to determine position of smart device in given venue. These knowledges have opened door to new possibilities for me.