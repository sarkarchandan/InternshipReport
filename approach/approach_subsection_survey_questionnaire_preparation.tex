\subsection{Survey Questionnaire Preparation}

Since Physical Web is a fairly new approach, at the very first phase of the mobility study, we attempted to see if at all we can reliably communicate with the students using smart devices. In order to do this, we planned on placing beacons in the university that send a URL that directs to a survey. The idea was to figure out the reliability of the beacons as well as to gather some insight on student mobility.

In the survey we focused on the following key points:
\begin{itemize}
	\item Mobility Pattern:
	\par The number of times a student needs to visit the designated campus per week or whether or not he/she needs to visit any other campuses.
	\item Transport: 
	\par The usual means of transport.
	\item Additional Features:
	\par The probable additional features that could make BLE beacon interaction useful for a student.
\end{itemize}

We prepared the survey pages both in German and English language since our target group of students could be both German and International. In additional we tried to keep the survey small and concise keeping in mind that the medium of interaction is going to be a smart device and students could feel de-motivated to fill out a long and tedious questionnaire.

Following are the snapshot of the survey Questionnaire that we have prepared.

\begin{figure}[H]
	\centering
	\includegraphics[width = \linewidth]{./images/survey_questions_de}
	\caption{Survey Questionnaire In German}
	\label{figure1:survey_german}
\end{figure}

And following is the English translation of the same
\begin{figure}[H]
	\centering
	\includegraphics[scale = 0.3]{./images/survey_questions_en}
	\caption{Survey Questionnaire English Translation}
	\label{figure1:survey_englishn}
\end{figure}

\par In the next section we are going to discuss how we published the survey online and made available for the students to interact with.
