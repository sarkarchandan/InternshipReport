\subsection{Unique Detections From Beacons}

Before going ahead with analysis of the dataset available with us we would like to summarize our approach so far. We have understood, once configured, a beacon placed in a location constantly emits a URL. Google Nearby and Physical Web application installed in smart devices scan for this signals and create interactive notifications. This detection events are recorded in the backend data store of Rakete 7 Substance\cite{Rakete7Substance} application. Rakete 7 made this data available for our mobility study in the form of either an API or a comma separated values dump. For our experiment at this phase we chose the later one.

\textbf{What is the structure of this dataset?}
\begin{itemize}
	\item \textbf{id:}
	\par Unique id for each of the instance of detection
	\item \textbf{tenant id:}
	\par Unique id for a potential customer who is availing the service offered by Rakete 7
	\item \textbf{access type:}
	\par Whether the particular detection took place via a beacon, channel or web. Channel and Web has Rakete 7 specific definitions and they are not relevant to our study. We, therefore focused only on those events generated by the beacons.
	\item \textbf{url:} 
	\par Unique id of the beacons placed in different designated locations. This field is significant for us because beacon ids are basically our reference for the different campus locations that we have included in our mobility study.
	\item \textbf{token:}
	\par This field is rather interesting and played a crucial role in detecting the uniqueness in data set. Values recorded in is a hash value generated from the users IP address. We understand that IP address of a particular device may change from network to network but apparently, this is the only field that refers to the users with their smart devices being detected via the beacons. Hence, we had to assume this field as an approximate reference for unique user.
	\item \textbf{poi id:}
	\par Unique id for the POI that we configured in the Content Management System of Rakete 7. We have configured only one POI and it refers to the url for our survey and the same POI configuration has been applied to all our four beacons.
	\item \textbf{user agent:}
	\par This field provides an approximate idea on the user device types that has been engaged in the beacon detection events
	\item \textbf{created at:}
	\par The time-stamp for a particular instance of detection dataset first created
	\item \textbf{updated at:}
	\par The time-stamp for a particular instance of the detection dataset last modified
\end{itemize}

Given the afore mentioned schema for this dataset, it is quite likely for a device to discover a given beacon repeatedly which results in repeated entry in Rakete 7 Substance for a given detection event for a a particular beacon. This means creation of multiple instance of detection for given user token (referring to a unique user as per our assumption), over a given period of time when that user with his/her smart device was in the proximity of the beacon.
Hence, we noticed that the dataset generated from the Substance have considerable amount of repetition that we would have to get rid of in order to get a unique dataset.

\textbf{Why is a unique data set required?}
\par The goal of our project is to evaluate the credibility of the Physical Web along with Bluetooth Low Energy beacon technology as a potential approach for a mobility study. The goal of a mobility study is to find out how many people are where at what time. Also, to get information about the movement behavior of the people. To achieve both goals it is inevitable to detect devices - and therefore approximately people - uniquely. Meaning, when a device is detected again we need be able to detect this. Only then we can get reliable about how many people are where and only then we can detect if one device moved from one beacon - and therefore a location - to another.
Also, to evaluate the reliability of the beacon technology for a mobility study, we need to know the percentage of how many devices that passed by a beacon actually detect the beacon. This percentage tells us the reliability of this tool.
\par \textbf{This means we have to keep the repeated entries for a given user token out of the equation, for a given location.} 

\textbf{Challenge before us:}
\par How to derive unique data set that could help us to determine a precise measure of the detections. Moreover, how to programmatically figure out a reusable model to get the location wise detection information on demand.

\textbf{Our Approach}
\par We wanted to achieve the following with our model:
\begin{itemize}
	\item To persist the data into database and determine the unique entries depending on given parameters using SQL.
	\item To expose the appropriate endpoint to query and visualize unique data.
	\item To make the project portable so that we could execute it without the underlying configurations overhead.
\end{itemize}

\textbf{Our Implementation:}
\par We  created a project and called it \texttt{DetectionApproximation} with the Java EE Specification. This project has dedicated modules responsible for the tasks. We would like to briefly describe the modules along with their responsibilities. 
\begin{itemize}
	
	\item \texttt{Persistence}: 
	\par This module is powered by Java Persistence API (JPA). It deals with Object Relational mapping between the Java Entity and corresponding  Relational Table in the PostgreSQL database.

	\item \texttt{Engine}: 
	\par This module is powered by Enterprise Java Beans (EJB). It encapsulates the methods for the following tasks:
	
		\subitem{-} \texttt{Persist Data}: The raw dataset has been fed into the project as a CSV file. This method parses the CSV file and creates the collection of POJO Entity objects and finally attempts to persist the same into the relational database.
		
		\subitem{-} \texttt{Detection For Given Beacon}: This method defines am SQL named native query and fetches the unique result set for a passed in beacon if from the database.
		
		\subitem{-} \texttt{Detection For Given User Token}: This method also defines a separate SQL named native query and fetches the unique result set for a passed in user token.
	Engine module has others helper methods as well which either uses the JPQL named queries or named native queries to manipulate the data.
	
	\item \texttt{Endpoint}: 
	\par This module is powered by Java API for RESTful Web Services (JAX-RS) specification along with its reference implementation named Jersey. This module has defined two distinct RESTful web services endpoints and corresponding HTTP GET methods to fire appropriate methods to return the result set as JSON response.
\end{itemize}

\textbf{Orchestrating things together...}
\par Finally, we have used Docker, more precisely Docker-Compose\cite{DockerCompose} utility to modularize our project into two dedicated containers and orchestrate between them. Briefly, 

\begin{figure}[H]
	\centering
	\includegraphics[height = 28mm]{./images/compose}
\end{figure}

\begin{itemize}
	\item \texttt{Glassfish}: 
	\par This container uses the \texttt{glassfish:latest} Docker image and provides a glassfish application server environment. Our project uses Gradle as the underlying dependency management and build/automation tool. Gradle builds the project and creates a Web Application Archive which is later deployed to the glassfish by Gradle itself.
	
	\item \texttt{PostgreSQL}: 
	\par This container uses the \texttt{postgres:latest} Docker image and creates PostgreSQL database server for us. We configure the PostgreSQL database with the help of relevant Environment Variables. 
\end{itemize}
Docker also helps us with necessary Glassfish server configurations enabling the successful deployment of the Web Application Archive generated for our project into the glassfish application server running inside the container. Gradle helps us with compilation and building the entire project and Docker makes the project portable and executable at any systems running Docker.

\textbf{How does our model work?}

While orchestrating using Docker-Compose in this sample execution, we have mapped the internal port 8080 of the Glassfish server container with the external port 8888. This will make our endpoints available at the this external port. We have exposed few endpoints where we can query for the total or unique result set. According to our discussion with Rakete 7 we have assumed that the \textbf{user token aka detectionToken in our model} will refer to the uniqueness for each user. That means each user will have a unique user token. 

However, this field is not globally unique because this field is derived from a hash value over the IP address of the smart device user is carrying and the IP address of the devices may change with the change in networks. After careful considerations, we have selected this field as an approximation of the uniqueness in terms of users because we have not found no other field more suitable. Following are a set of experiments that were made before preparing the report.

\begin{itemize}
	\item \texttt{Total Result Set}: This endpoint will expose a list of all detections just in case we want to have a sneak peek on the  raw detection events.We query the  endpoint:
		\begin{verbatim}
		http://localhost:8888/endpoint/rest/stronger
		\end{verbatim}
		\begin{figure}[H]
			\centering
			\includegraphics[width = \linewidth]{./images/all_detections}
			\caption{All Detection Events Recorded}
			\label{figure1:all_detection_events_recorded}
		\end{figure}
	This endpoint is useful for comparing the no of unique detection events against the total no of detection events which in our case ins 497.
		
	\item \texttt{Location-wise Unique Result Set}: With this endpoint, we use to get the location wise unique detection events. Since we have deployed the beacons, we have kept track of which beacon has been placed at which location along with respective beacon ids. For instance, the Beacon placed in Cafeteria of the Feki campus  has the id 8ybr0kz0. We can place this id as a query parameter in this end point and get all the unique detection events for this beacon.
	We query the endpoint:
	\begin{verbatim}
	http://localhost:8888/endpoint/rest/detection_beacon?beacon=8ybr0kz0
	\end{verbatim}
	\begin{figure}[H]
		\centering
		\includegraphics[width = \linewidth]{./images/unique_detection_feki_mensa}
		\caption{Unique Detections At Feki Campus Cafeteria}
		\label{figure1:unique_detections_at_feki_campus_mensa}
	\end{figure}
	We can notice that the number of unique detections made by this beacon is 11. Similarly, we have derived all the unique detection result sets for rest of the three locations as well. As per our findings, total Unique Detections made by all four Beacons are 33. We will discuss on this figures more in our reports going forward.
	
	\item \texttt{User Token-wise Unique Result Set}: This endpoint is rather interesting. It will be handy if we want to check the mobility pattern for a given user token. Let's have this example where we are trying to take a look at the mobility pattern for the user identified by the user token \texttt{907c13d9803ff59910c7bd70d29343b2}.
	\begin{verbatim}
	http://localhost:8888/endpoint/rest/detection_user
	?user=907c13d9803ff59910c7bd70d29343b2
	\end{verbatim}
	\begin{figure}[H]
		\centering
		\includegraphics[width = \linewidth]{./images/unique_detection_user}
		\caption{Mobility Pattern For a Given User}
		\label{figure1:mobility_pattern_for_given_user}
	\end{figure}
	From this experiment we can have an approximate idea of the mobility pattern for this user token. From this unique dataset it is evident that this person whom this user token refers to was at the following locations at the respective time frame.
	\begin{itemize}
		\item Feki Mensa (Beacon Id: 8ybr0kz0) around the time frame of 2017-07-10 10:27:20
		\item Markushaus (Beacon Id: k1pto67w) around the time frame of 2017-07-05 13:43:28
		\item ERBA 02.008 (Beacon Id: ta9cguz5) around the time frame of 2017-07-07 09:20:09
		\item ERBA Mensa (Beacon Id: unz3xmjb) around the time time frame of 2017-07-05 10:15:17
	\end{itemize}
	This data could be useful in studying approximately how many students have moved between what locations at what time frame.
	
	\par Since we now have rather a strong notion of detection, we would like to move to yet another stone yet to be turned. Our goal dictates us to explore the possibility of using MAC addresses in our mobility study. Goal is to find out the vendor specific information from the MAC address if MAC address itself has been made available. In the upcoming section we have presented our study in this area.
	
\end{itemize}






   

