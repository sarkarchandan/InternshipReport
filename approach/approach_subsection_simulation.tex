\subsection{Student Mobility Simulation }

Beacon technology and Bluetooth LE is prognosed to be come a major trend in the years to come \cite{venzke2014standortlokalisierung} \cite{beaconPrognose}. By now the awareness for beacons has not yet arrived by the majority of people.  Also, technological requirements are not met by most people. A smartphone is needed with turned on Bluetooth and an activated "nearby" functionality or app in order to be able to detect beacons. Even if a beacon is detected by a smartphone one still does not know if the person interacts with the beacon and clicks the link. 
\par Starting from the status quo of beacon detections and interactions in the university different scenarios are imaginable for the future. In order to explore them and find out how many students could and would detect and interact with beacons in the future a simulation has been built. Different variables can be changed and therefore different future scenarios can be simulated that will help to estimate future beacon usage.
\par For the simulation we used the open source NetLogo software (cf. chapter 3 Background). In the 2D spatial view we simulated the ERBA building of the Otto-Friedrich University. Two different agent breeds are used: locations and students. As locations entrances, lecture rooms, the cafeteria and the library are marked. Students are moving through the ERBA building after different set of rules. 
\par In the simulation there are beacons placed at different locations: The cafeteria, the foyer entrance, the entrance to the building with the seminar rooms and the big lecture room. Detections and interactions of the students with beacons are counted. Whether a student detects and interacts with a beacon is determined by different variables.
\par In order to get an accurate model the input of the data needs to be realistic. Therefore two different inputs needed to be configured: 
\begin{itemize}
\item How many students are where at what time?
\item What are relevant variables for beacon detection and to what value do we have to set those variables?
\end{itemize}

\paragraph{Data Collection}
\par Student movement data was collected with different approaches. Data was gathered from University's UnivIS to determine at what time how many students should be in which rooms. Data from the cafeteria's point-of-sale system could be used to determine how many students usually are in the cafeteria at different times. This data can be provided by the Studentenwerk Würzbug, but they need some time to generate it, so it couldn't be used in the current simulation. The project group "Data Quality" provided us with information about how many students are in the foyer entrance at what time, using the data from the flow tracker, which is installed in the foyer of the erba building. The "Data Quality" group could also provide the MAC-adresses from the flow tracker so it could be compaird how many devices see the flow tracker and how many of them also see the beacon. All this data could be imported and used by the simulation or could be used to evaluate it and adjust the parameters.

\par After consultation with Rakete7 different variables were identified that build the probability of detecting beacons:
\begin{itemize}
\item \textbf{iPhone vs. Android Smartphone:} Android smartphones with Android 5.0 or higher support nearby natively \cite{googleNearby}. iPhones need an App (e.g. Physical Web App \cite{physicalWebApp})to be able to detect Beacons. Therefore, Android smartphone owners are more likely to detect beacons because no prior app download is needed. Also, it is likely that the majority of iPhone users does not have such an app installed yet. If the Physical Web is reaching more awareness in the future, if more beacons are distributed and more people know about the technology the usage of such apps is likely to increase.
\par In Germany the market share of iOS is about 17\% compared to 80\% sold Android devices \cite{iosVsAndroid}. About 74\% of the Android devices run on Android 5.0 or higher \cite{androidPlatformVersions}. Based on experience with past Android versions one can say that it takes less than 1.5 years for a version to rise from 70\% of people who at least have that version to 90\%. It takes another year to reach the 95\%. The last 5\% take about another 2 years. Therefore one can safely assume, that within 3 years about 95\% of the devices will at least run on Android 5.0 or higher \cite{androidHistorical}. 
\item \textbf{Active Bluetooth:} Beacons can only be detected if the device's Bluetooth LE is turned on. Some devices always keep BLE turned on, even if the regular Bluetooth is turned off. Unfortunately, we could not find any statistic on how many devices have that functionality or how this is supposed to develop in the future.
\par According to the Mobile Communications Report 2016 which was a study conducted in Austria 79\% of the participants are using Bluetooth. 23\% of them have the Bluetooth function always turned on \cite{bluetoothUsage}. It is likely that Bluetooth usage will rise in the future since more and more functions work via Bluetooth (e.g. iPhone7 has no more AUX connection but recommends Bluetooth for speakers). Therefore, many users might not turn their Bluetooth off anymore in the future.
\item \textbf{Technical Detection Probability:} Even if the smartphone fits the necessary preconditions there might be other technical reasons, why a beacon is not detected. For example due to random errors or if the student passes the beacon too fast.
\end{itemize} 
Once a beacon is detected the student still has to interact with the received notification. We included the variable if the student just passes the beacon or if he or she is staying in a room where the beacon is placed. Our hypothesis is, that the interaction probability is higher in a dwelling situation. Other variables that influence the interaction probability are not known yet. Therefore, we decided to work in our model just with a general interaction probability that is split into a 'sitting interaction probability' and 'walking interaction probability'. Data for the detection-interaction ratio is drawn from our experiences with the deployed beacons.